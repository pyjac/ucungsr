\subsection{Hamming Window}
\index{Hamming Window}

$Revision: 1.13 $

\subsubsection{Implementation}
\index{Hamming Window!Implementation}

The Hamming Window implementation in {\marf} is in the
\api{marf.math.Algortithms.Hamming} class as of
version 0.3.0-devel-20050606 (a.k.a 0.3.0.2).

\subsubsection{Theory}
\index{Hamming Window!Theory}

In many DSP techniques, it is necessary to consider a smaller portion of the
entire speech sample rather than attempting to process the entire sample at
once.  The technique of cutting a sample into smaller pieces to be considered
individually is called ``windowing''.  The simplest kind of window to use is
the ``rectangle'', which is simply an unmodified cut from the larger sample.

$$
r(t) =
\left\{
{
	\begin{array}{ll}
		1 & \mbox{ for } (0 \le t \le N-1) \\
		0 & \mbox{ otherwise }
	\end{array}
}
\right.
$$

Unfortunately, rectangular windows can introduce errors, because near the edges
of the window there will potentially be a sudden drop from a high amplitude
to nothing, which can produce false ``pops'' and ``clicks'' in the analysis.

A better way to window the sample is to slowly fade out toward the edges, by
multiplying the points in the window by a ``window function''.  If we take
successive windows side by side, with the edges faded out, we will distort
our analysis because the sample has been modified by the window function.
To avoid this, it is necessary to overlap the windows so that all points in
the sample will be considered equally.  Ideally, to avoid all distortion, the
overlapped window functions should add up to a constant.  This is exactly what
the Hamming window does.  It is defined as:

$$ x = 0.54 - 0.46 \cdot \cos\left(\frac{2 \pi n}{l-1}\right) $$

\noindent
where $x$ is the new sample amplitude, $n$ is the index into the window, and $l$ is
the total length of the window.

% EOF
