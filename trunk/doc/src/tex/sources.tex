\section{MARF Source Code}
\index{MARF!Source Code}

$Revision: 1.8 $

\subsection{Project Source and Location}
\index{MARF!Project Location}

Our project since the its inception has always been an open-source project.
All releases including the most current one should most of the time be
accessible via \verb+<http://marf.sourceforge.net>+ provided by \texttt{SourceForge.net}.
We have a complete API documentation as well as this manual and all the sources
available to download through this web page.

\subsection{Formatting}
\index{MARF!source code formatting}
\index{Source code formatting}

Source code formatting uses a 4 column tab spacing, currently with
tabs preserved (i.e. tabs are not expanded to spaces).

For Emacs, add the following (or something similar)
to your \file{\~/.emacs}
initialization file:

\begin{verbatim}
;; check for files with a path containing "marf"
(setq auto-mode-alist
  (cons '("\\(marf\\).*\\.java\\'" . marf-java-mode)
        auto-mode-alist))
(setq auto-mode-alist
  (cons '("\\(marf\\).*\\.java\\'" . marf-java-mode)
        auto-mode-alist))

(defun marf-java-mode ()
  ;; sets up formatting for MARF Java code
  (interactive)
  (java-mode)
  (setq-default tab-width 4)
  (java-set-style "bsd")      ; set java-basic-offset to 4, etc.
  (setq indent-tabs-mode t))  ; keep tabs for indentation
\end{verbatim}

For \tool{vim}, your
\file{\~/.vimrc} or equivalent file should contain
the following:

\begin{verbatim}
set tabstop=4
\end{verbatim}

    or equivalently from within vim, try

\begin{verbatim}
:set ts=4
\end{verbatim}

    The text browsing tools \tool{more} and
    \tool{less} can be invoked as

\begin{verbatim}
more -x4
less -x4
\end{verbatim}


\subsection{Coding and Naming Conventions}
\index{MARF!Coding Conventions}
\index{Coding and Naming Conventions}

For now, please see \texttt{http://marf.sf.net/coding.html}.

{\todo}

% EOF
