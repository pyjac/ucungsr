\subsection{TestFilters}
\index{Applications!TestFilters}
\index{Testing Applications!TestFilters}

$Revision: 1.6 $

\api{TestFilters} is one of the testing applications in {\marf}. It tests how four types of
FFT-filter-based preprocessors work with simulated or real wave type sound samples.
From user's point of view, \api{TestFilters} provides with usage, preprocessors, and loaders
command-line options.
By entering \texttt{--help}, or \texttt{-h}, or
when there are no arguments, the usage information will be displayed.
It explains the arguments used in \api{TestFilters}. The first argument is the type of
preprocessor/filter. These are high-pass filter (\texttt{--high}), low-pass filter (\texttt{--low});
band-pass filter (\texttt{--band}); high frequency boost preprocessor (\texttt{--boost})
to be chosen. Next argument is the type of loader to use to load initial testing sample.
\api{TestFilters} uses two types of
loaders, \api{SineLoader} and \api{WAVLoader}.
Users should enter either \texttt{--sine} or \texttt{--wave} as the second argument
to specify the desired loader. The argument \texttt{--sine} uses \api{SineLoader}
that will generate a plain sine wave to be fed to the selected preprocessor.
While the \texttt{--wave} argument uses \api{WAVLoader} to load a
real wave sound sample. In the latter case, users need to input the third argument --
sample file in the WAV format to feed to
\api{WAVLoader}. After selecting all necessary arguments, user can run and get the output of
\api{TestFilters} within seconds.

\noindent
Complete usage information:

\vspace{15pt}
\hrule
\begin{verbatim}

Usage:

    java TestFilters PREPROCESSOR LOADER [ SAMPLEFILE ]

    java TestFilters --help | -h
        displays this help and exits

    java TestFilters --version
        displays application and MARF versions

Where PREPROCESSOR is one of the following:
    --high           use high-pass filter
    --band           use band-pass filter
    --low            use low-pass filter
    --boost          use high-frequency-boost preprocessor
    --highpassboost  use high-pass-and-boost preprocessor

Where LOADER is one of the following:
    --sine    use SineLoader (generated sine wave)
    --wave    use WAVLoader; requires a SAMPLEFILE argument

\end{verbatim}

\hrule
\vspace{15pt}

The application is made to exercise the following {\marf} modules:

The main are the FFT-based filters in the
\api{marf.Preprocessing.FFTFilter.*} package.

\begin{enumerate}
\item
\api{BandpassFilter}
\item
\api{HighFrequencyBoost}
\item
\api{HighPassFilter}
\item
\api{LowPassFilter}
\end{enumerate}

Additionally, some other units were employed in this application:

\begin{enumerate}
\item
\api{marf.MARF}
\item
\api{marf.util.OptionProcessor}
\item
\api{marf.Storage.Sample}
\item
\api{marf.Storage.SampleLoader}
\item
\api{marf.Storage.WAVLoader}
\item
\api{marf.Preprocessing.Preprocessing}
\end{enumerate}

The main \api{MARF} module enumerates
these preprocessing modules as \api{HIGH\_FREQUENCY\_BOOST\_FFT\_FILTER}, \api{BANDPASS\_FFT\_FILTER}, \api{LOW\_PASS\_FFT\_FILTER}, and
\api{HIGH\_PASS\_FFT\_FILTER}, and incoming sample file format as \api{WAV}, \api{SINE}. \api{OptionProcessor} helps maintaining and
validating command-line options. \api{Sample} maintains incoming and processed sample data. \api{SampleLoader} provides sample loading
interface for all the MARF loaders. It must be overridden by a concrete sample loader such as \api{SineLoader} or \api{WAVLoader}.
\api{Preprocessing} does
general preprocessing such as \api{preprocess()} (overridden by the filters),
\api{normalize()}, \api{removeNoise()} and \api{removeSilence()} out of which
for this application the former two are used.
In the end, above modules work together to test the work of the filters and produce
the output to STDOUT.
The output of \api{TestFilters} is the filtered data from the original signal fed to each
of the preprocessors. It provides both users and programmers internal
information of the effect of MARF preprocessors so they can be compared with the
expected output in the \file{expected} folder to detect any errors if the underlying
algorithm has been changed.
