\subsection{Min/Max Amplitudes}
\label{sect:minmax}
\index{Min/Max Amplitudes}
\index{Feature Extraction!Min/Max Amplitudes}

$Revision: 1.4 $

\subsubsection{Description}

The Min/Max Amplitudes extraction simply involves
picking up $X$ maximums and $N$ minimums out of the
sample as features. If the length of the sample
is less than $X+N$, the difference is filled in
with the middle element of the sample.

TODO: This feature extraction does not perform very well yet
in any configuration because of the simplistic implementation:
the sample amplitudes are sorted and $N$ minimums and $X$ maximums
are picked up from both ends of the array. As the samples are
usually large, the values in each group are really close if
not identical making it hard for any of the classifiers
to properly discriminate the subjects. The future improvements
here will include attempts to pick up values in $N$ and $X$
distinct enough to be features and for the samples smaller than the $X+N$ sum,
use increments of the difference of smallest maximum and
largest minimum divided among missing elements in the middle
instead one the same value filling that space in.

% EOF
