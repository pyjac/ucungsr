\chapter{Introduction}
\index{MARF!Introduction}

$Revision: 1.22 $

\section{What is MARF?}

{\marf} stands for {\bf M}odular {\bf A}udio {\bf R}ecognition {\bf F}ramework.
It contains a collection of algorithms for Sound, Speech, and Natural Language
Processing arranged into an uniform framework to facilitate addition of new
algorithms for preprocessing, feature extraction, classification, parsing, etc.
implemented in Java.
{\marf} is also a research platform for various performance metrics of the
implemented algorithms.

\subsection{Purpose}
\index{MARF!Purpose}

Our main goal is to build a general open-source framework to allow developers in the
audio-recognition industry (be it speech, voice, sound, etc.) to choose and apply various methods,
contrast and compare them, and use them in their applications. As a proof of concept, a
user frontend application for Text-Independent (TI) Speaker Identification has
been created on top of the framework (the \api{SpeakerIdentApp} program). A variety
of testing applications and applications that show how to use various aspects of {\marf}
are also present. A new recent addition is some (highly) experimental NLP support, which is also
included in {\marf} as of 0.3.0-devel-20050606 (0.3.0.2). For more information on applications
that employ {\marf} see \xc{chapt:apps}.

\subsection{Why {\java}?}

We have chosen to implement our project using the Java programming language. This
choice is justified by the binary portability of the Java applications as well as
facilitating memory management tasks and other issues, so we can concentrate more on
the algorithms instead. Java also provides
us with built-in types and data-structures to manage
collections (build, sort, store/retrieve) efficiently \cite{javanuttshell}.

\subsection{Terminology and Notation}

$Revision: 1.4 $

The term ``{\marf}'' will be
to refer to the software that accompanies this
documentation.
An {\bf application programmer}
could be anyone who is using, or wants to use, any part of the
{\marf} system.
A {\bf {\marf} developer}
is a core member of {\marf} who is hacking away the
{\marf} system.

% EOF


\section{Authors, Contact, and Copyright Information}
\index{MARF!Contact}
\index{MARF!Copyright}

\subsection{COPYRIGHT}

$Revision: 1.11 $

{\marf} is Copyright $\copyright$ 2002 - 2006
by the MARF Research and Development Group and is distributed under
the terms of the BSD-style license below.

\vspace{15pt}
\noindent
Permission to use, copy, modify, and distribute this software and
its documentation for any purpose, without fee, and without a
written agreement is hereby granted, provided that the above
copyright notice and this paragraph and the following two paragraphs
appear in all copies.

\vspace{15pt}
\noindent
IN NO EVENT SHALL CONCORDIA UNIVERSITY OR THE AUTHORS BE LIABLE TO ANY
PARTY FOR DIRECT, INDIRECT, SPECIAL, INCIDENTAL, OR CONSEQUENTIAL
DAMAGES, INCLUDING LOST PROFITS, ARISING OUT OF THE USE OF THIS
SOFTWARE AND ITS DOCUMENTATION, EVEN IF CONCORDIA UNIVERSITY OR THE AUTHORS
HAVE BEEN ADVISED OF THE POSSIBILITY OF SUCH DAMAGE.

\vspace{15pt}
\noindent
CONCORDIA UNIVERSITY AND THE AUTHORS SPECIFICALLY DISCLAIM ANY WARRANTIES,
INCLUDING, BUT NOT LIMITED TO, THE IMPLIED WARRANTIES OF
MERCHANTABILITY AND FITNESS FOR A PARTICULAR PURPOSE.  THE SOFTWARE
PROVIDED HEREUNDER IS ON AN ``AS-IS'' BASIS, AND CONCORDIA UNIVERSITY AND THE
AUTHORS HAVE NO OBLIGATIONS TO PROVIDE MAINTENANCE, SUPPORT,
UPDATES, ENHANCEMENTS, OR MODIFICATIONS.


\subsection{Authors}
\index{MARF!Authors}

\noindent
Authors Emeritus, in alphabetical order:
\index{MARF!Authors Emeritus}

\begin{itemize}
	\item Ian Cl\'ement, \url{iclement@users.sourceforge.net}
	\item Serguei Mokhov, \url{mokhov@cs.concordia.ca}, a.k.a Serge
	\item Dimitrios Nicolacopoulos, \url{pwrslave@users.sourceforge.net}, a.k.a Jimmy
	\item Stephen Sinclair, \url{radarsat1@users.sourceforge.net}, a.k.a. Steve, radarsat1
\end{itemize}

\noindent
Contributors:
\index{MARF!Contributors}

\begin{itemize}
	\item Shuxin Fan, \url{fshuxin@gmail.com}, a.k.a Susan
\end{itemize}

\noindent
Current maintainers:
\index{MARF!Current maintainers}

\begin{itemize}
	\item Serguei Mokhov, \url{mokhov@cs.concordia.ca}
\end{itemize}

\noindent
If you have
some suggestions, contributions to make, or for bug reports, don't
hesitate to contact us :-)
For {\marf}-related issues please contact us at \url{marf-devel@lists.sf.net}.
Please report bugs to \url{marf-bugs@lists.sf.net}.

%
%
%

\section{Brief History of MARF}
\index{MARF!History}

$Revision: 1.13 $

The {\marf} project was initiated in September 26, 2002 by four students of
Concordia University, Montr\'eal, Canada as their course project for Pattern Recognition under
guidance of Dr. C.Y. Suen. This included Ian Cl\'ement, Stephen Sinclair,
Jimmy Nicolacopoulos, and Serguei Mokhov.

\subsection{Developers Emeritus}

\begin{itemize}
\item
Ian's primary contributions were the
LPC and Neural Network algorithms support with the \api{Spectrogram} dump.

\item
Steve has done an extensive
research and implementation of the FFT algorithm for feature extraction
and filtering and Euclidean Distance with the \api{WaveGrapher} class.

\item
Jimmy was focused on implementation of the WAVE file format
loader and other storage issues.

\item
Serguei designed the entire MARF framework and architecture,
originally implemented general distance classifier and its Chebyshev, Minkowski,
and Mahalanobis incarnations along with normalization of the sample data. Serguei
designed the Exceptions Framework of MARF and was involved into the integration
of all the modules and testing applications to use {\marf}.
\end{itemize}

\subsection{Contributors}

\begin{itemize}
\item
	Shuxin `Susan' Fan contributed to development and maintenance of some test applications
	(e.g. \api{TestFilters}) and initial adoption of the JUnit framework \cite{junit} within {\marf}.
	She has also finalized some utility modules (e.g. \api{marf.util.Arrays})
	till completion and performed {\marf} code auditing and ``inventory''.
	Shuxin has also added NetBeans project files to the build system of {\marf}.

\end{itemize}

\subsection{Current Status and Past Releases}

Now it's a project on its own, being maintained and developed as we have some spare time for it.
When the course was over, Serguei Mokhov is the primary maintainer of the
project. He rewrote \api{Storage} support and polished all of {\marf} during
two years and added various utility modules and NLP support and implementation of new
algorithms and applications. Serguei maintains this manual, the web site and most of
the sample database collection. He also made all the releases of the project
as follows:

\begin{itemize}
\item 0.3.0-devel-20060226 (0.3.0.5), Sunday, February 26, 2006
\item 0.3.0-devel-20050817 (0.3.0.4), Wednesday, August 17, 2005
\item 0.3.0-devel-20050730 (0.3.0.3), Saturday, July 30, 2005
\item 0.3.0-devel-20050606 (0.3.0.2), Monday, June 6, 2005
\item 0.3.0-devel-20040614 (0.3.0.1), Monday, June 14, 2004
\item 0.2.1, Monday, February 17, 2003
\item 0.2.0, Monday, February 10, 2003
\item 0.1.2, December 17, 2002 - Final Project Deliverable
\item 0.1.1, December 8, 2002 - Demo
\end{itemize}

\noindent
The project is currently geared towards completion planned TODO items on {\marf} and
its applications.

% EOF


%
%
%

\section{MARF Source Code}
\index{MARF!Source Code}

$Revision: 1.8 $

\subsection{Project Source and Location}
\index{MARF!Project Location}

Our project since the its inception has always been an open-source project.
All releases including the most current one should most of the time be
accessible via \verb+<http://marf.sourceforge.net>+ provided by \texttt{SourceForge.net}.
We have a complete API documentation as well as this manual and all the sources
available to download through this web page.

\subsection{Formatting}
\index{MARF!source code formatting}
\index{Source code formatting}

Source code formatting uses a 4 column tab spacing, currently with
tabs preserved (i.e. tabs are not expanded to spaces).

For Emacs, add the following (or something similar)
to your \file{\~/.emacs}
initialization file:

\begin{verbatim}
;; check for files with a path containing "marf"
(setq auto-mode-alist
  (cons '("\\(marf\\).*\\.java\\'" . marf-java-mode)
        auto-mode-alist))
(setq auto-mode-alist
  (cons '("\\(marf\\).*\\.java\\'" . marf-java-mode)
        auto-mode-alist))

(defun marf-java-mode ()
  ;; sets up formatting for MARF Java code
  (interactive)
  (java-mode)
  (setq-default tab-width 4)
  (java-set-style "bsd")      ; set java-basic-offset to 4, etc.
  (setq indent-tabs-mode t))  ; keep tabs for indentation
\end{verbatim}

For \tool{vim}, your
\file{\~/.vimrc} or equivalent file should contain
the following:

\begin{verbatim}
set tabstop=4
\end{verbatim}

    or equivalently from within vim, try

\begin{verbatim}
:set ts=4
\end{verbatim}

    The text browsing tools \tool{more} and
    \tool{less} can be invoked as

\begin{verbatim}
more -x4
less -x4
\end{verbatim}


\subsection{Coding and Naming Conventions}
\index{MARF!Coding Conventions}
\index{Coding and Naming Conventions}

For now, please see \texttt{http://marf.sf.net/coding.html}.

{\todo}

% EOF


%
%
%

\section{Versioning}
\index{Versioning}
\index{MARF!Versioning}

This section attempts to clarify versioning scheme employed
by the {\marf} project for stable and development releases.

In the 0.3.0 series a four digit version number was introduced
like 0.3.0.1 or 0.3.0.2 and so on. The first digit indicates
a {\em major version}. This typically indicates a significant
coverage of implementations of major milestones and improvements,
testing and quality validation and verification to justify a major
release. The {\em minor version} has to do with some smaller
milestones achieved throughout the development cycles.
It is a bit subjective of what the minor and major version bumps are, but
the TODO list in \xa{appx:todo} sets some tentative milestones
to be completed at each minor or major versions. The {\em revision}
is the third digit is typically applied to stable releases if there
are a number of critical bug fixes in the minor release, it's revision
is bumped. The last digit represents the {\em minor revision} of the
code release. This is typically used throughout development releases
to make the code available sooner for testing. This notion was first
introduced in 0.3.0 to count sort of increments in {\marf} development,
which included bug fixes from the past increment and some chunk of new
material. Any bump of major, minor versions, or revision, resets the
minor revision back to zero. In the 0.3.0-devel release series these
minor revisions were publicly displayed as dates (e.g. 0.3.0-devel-20050817)
on which that particular release was made.

All version as of 0.3.0 can be programmatically queried for
and validated against. In 0.3.0.5, a new \api{Version} class
was introduced to encapsulate all versioning information and
some validation of it to be used by the applications.

\begin{itemize}
\item
	the major version can be obtained from \api{marf.MARF.MAJOR\_VERSION}
	and \api{marf.Version.MAJOR\_VERSION} where as of 0.3.0.5 the former
	is an alias of the latter

\item
	the minor version can be obtained from \api{marf.MARF.MINOR\_VERSION}
	and \api{marf.Version.MINOR\_VERSION} where as of 0.3.0.5 the former
	is an alias of the latter

\item
	the revision can be obtained from \api{marf.MARF.REVISION}
	and \api{marf.Version.REVISION} where as of 0.3.0.5 the former
	is an alias of the latter

\item
	the minor revision can be obtained from \api{marf.MARF.MINOR\_REVISION}
	and \api{marf.Version.MINOR\_REVISION} where as of 0.3.0.5 the former
	is an alias of the latter
\end{itemize}

The \api{marf.Version} class provides some API to validate the version
by the application and report mismatches for convenience. See the API
reference for details. One can also query a full \api{marf.jar} or 
\api{marf-debug.jar} release for version where all four components
are displayed by typing:

\begin{verbatim}
    java -jar marf.jar --version
\end{verbatim}

% EOF
